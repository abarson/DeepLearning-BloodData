\documentclass{article}
\usepackage{amssymb}
\usepackage{amsmath}
\usepackage{setspace}
\usepackage{graphicx}
\usepackage{algorithm}
\usepackage{algorithmicx}
\usepackage[noend]{algpseudocode}

%\usepackage[margin=1in]{geometry}
\title{CS395 Spring 2018---Final Project Report \\ \large Heart Rate Prediction with Deep Video Regression}
\author{
  Adam Barson \\ \small{\texttt{abarson@uvm.edu}}
  \and Daniel Berenberg \\ \small{\texttt{djberenb@uvm.edu}
  \thanks{in collaboration with Dr. Ryan McGinnis, UVM} 
 \thanks{under advisory of Dr. Safwan Wsah, UVM}}
}
\date{\today}
\onehalfspace
\begin{document}
\maketitle
\section*{Introduction}
Millions of Americans are afflicted with panic disorder [1], a psychiatric disorder in which debilitating fear and anxiety arise with no apparent cause [2]. There are several clinically available methods to treat panic disorder, many of which involve either medication or intensive psychotherapy [3]. Past research [4] in the biomedical field has shown that by simply showing a panic disorder victim their heart rate on the onset of or during a panic attack, their episode was significantly mitigated or weakened in intensity. \\ \\
Allowing people afflicted with panic disorder to access vitals such as heart rate and respiratory rate could not only benefit the longer term management of the disorder, but mitigate the risks and side effects of a live panic attack. \\ \\
In order to expose this treatment method to as many victims of panic disorder as possible, we explore an element of the solution by making use of the ubiquitous smartphone. There is promising evidence that suggests pulse is detectable by processing videos in smartphone cameras. By pressing a finger against a smartphone camera with the flashlight activated, we can obtain a highly resolved clip of the blood pulsating within the finger. This project aims to leverage deep learning techniques in order to predict a patient?s heart rate from such a video. \\ \\

\end{document}